\documentclass{article} 

\usepackage{color}
\usepackage{amsmath}
\usepackage{german}
\usepackage[utf8]{inputenc}

\begin{document}

\begin{center}
{\Large Python Testumgebung}\\[3ex]
{\bf Wie sehen diese Matrizen aus?}
\end{center}
\bigskip


\[A=
\begin{pmatrix}
  0& 0& 4& 12& 5\\
  3& 1 &4 &13& 12\\
  0& 7& 3& 6& 7\\
  5& -1& 0& -1& 11\\
  10& 2& 6 &7 &12\\
  12& 11& 2& 11& 11
\end{pmatrix} 
B=
\begin{pmatrix}
11& 7& 3& 12& 5& 10\\
6& 13& 10 &6 &8 &9\\
-1& 6 &-1 &3 &10& 1\\
11& 3& 4& 7& 3& 6\\
7& 7 &5 &12& 7& 7  
\end{pmatrix}\]
\begin{center}
  Wird zerlegt in:
  \end{center}
\[A_{11}=
\begin{pmatrix}
  0& 0& 4&\\
  3& 1 &4 \\
  0& 7& 3&
\end{pmatrix}
A_{12}=
\begin{pmatrix}
12& 5\\
13& 12\\
6& 7
\end{pmatrix}
A_{21}=
\begin{pmatrix}
5& -1& 0\\
10& 2& 6\\
12& 11& 2
\end{pmatrix}
A_{22}=
\begin{pmatrix}
-1& 11\\
7& 12\\
11& 11
\end{pmatrix}\]

\[B_{11}=
\begin{pmatrix}
11& 7& 3\\
6& 13& 10\\
-1& 6& -1
\end{pmatrix}
B_{12}=
\begin{pmatrix}
12& 5& 10\\
6& 8&  9\\
3& 10& 1
\end{pmatrix}
B_{21}=
\begin{pmatrix}
11& 3& 4\\
7& 7& 5
\end{pmatrix}
B_{22}=
\begin{pmatrix}
7& 3& 6\\
12& 7& 7
\end{pmatrix}\]  

\[A\cdot B=C=
\begin{pmatrix}
  A_{11}\cdot B_{11} + A_{12}\cdot B_{21} & A_{11}\cdot B_{12} + A_{12}\cdot B_{22}\\
  A_{21}\cdot B_{11} + A_{22}\cdot B_{12} & A_{21}\cdot B_{12} + A_{22}\cdot B_{22}
  \end{pmatrix}\]

\end{document}			                      
